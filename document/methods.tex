\section{Methods}
\subsection{Optimal \(\epsilon\) selection}
To choose the optimal value of \(\epsilon\) for each algorithm,
it must first be verified that the parameter is correlated
with the jointspace distance traveled, \(d_J\).
As the environment is constrained in that there is a minimum \(d_J\)
that the robot must travel, the distribution of \(d_J\) for each
value of \(\epsilon\) is expected have positive skew,
rather than being a normal distribution.
Due to the nature of the RRT algorithm, it is also expected that
the choice of \(\epsilon\) is not trivial;
Too large a value will make the robot move in too large steps,
and too small a value will make the random tree branch in
undesirable directions, both cases resulting in a large, suboptimal \(d_J\).
Thus, the relation between \(\epsilon\) and \(d_J\) can not be expected to be linear.
The correlation between \(\epsilon\) and \(d_J\) is tested using
Spearman's Rank Correlation, with the null-hypothesis that \(d_J\)
is uncorrelated with \(\epsilon\).

If correlation is shown, the relationship
between \(\epsilon\) and \(d_J\) can be used
to select an optimal \(\epsilon\) so that
\(d_J\) is minimized.
Due to the positive skew in the distribution
of \(d_J\), the selection is based on the median
of \(d_J\).
This means the optimal \(\epsilon\) is selected
according to the collected data,
as the one which yielded the smallest median of \(d_J\).