\section{Methods}
\subsection{Optimal \(\epsilon\) selection}
\(d_J\) was sampled \(n_1=100\) times for each algorithm,
for each \(\epsilon = 0.05, 0.15, 0.25, ..., 6.45\).

To choose the optimal value of \(\epsilon\) for each algorithm,
it must first be verified that the parameter is correlated
with the jointspace distance traveled, \(d_J\).
As the environment is constrained in that there is a minimum \(d_J\)
that the robot must travel, the distribution of \(d_J\) for each
value of \(\epsilon\) is expected have positive skew,
rather than being a normal distribution.
Due to the nature of the RRT algorithm, it is also expected that
the choice of \(\epsilon\) is not trivial;
Too large a value will make the robot move in too large steps,
and too small a value will make the random tree branch in
undesirable directions, both cases resulting in a large, suboptimal \(d_J\).
Thus, the relation between \(\epsilon\) and \(d_J\) can not be expected to be linear.
The correlation between \(\epsilon\) and \(d_J\) is tested using
Spearman's Rank Correlation, with the null-hypothesis that \(d_J\)
is uncorrelated with \(\epsilon\).

If correlation is shown, the relationship
between \(\epsilon\) and \(d_J\) can be used
to select an optimal \(\epsilon\) so that
\(d_J\) is minimized.
Due to the positive skew in the distribution
of \(d_J\), the selection is based on the median
of \(d_J\).
This means the optimal \(\epsilon\) is selected
according to the collected data,
as the one which yields the smallest median of \(d_J\).

\subsection{Algorithm comparison}
The algorithms are compared on \(d_J\),
given the optimal values \(\epsilon_{bi}\) and \(\epsilon_{ba}\)
of the bidirectional and balanced bidirectional RRT algorithms respectively.
\(d_J\) was sampled \(n_2=1000\) times for each algorithm.
The correct method of comparison must be based
on the distributions of \(d_J\) for the different algorithms.
These distributions may be assumed to be positively skewed.
They may not be assumed to have equal variancer.
The distributions are transformed using the log-transform,
to make the normal distribution assumption valid.
This assumption is verified by inspection of the histograms
and QQ-plots of each algorithm data set.
A normality test is also carried out using the Shapiro Wilks Test.
This test is expected to imply non-normality even for small
non-normalities with large sample sizes. In contrast,
large sample sizes such as \(n_2\), make the violation of the normality assumption
less important due to the central limit theorem.

The log-transformed distributions are tested for equal variance
using Fisher's F-test.
As the distributions do not have the same variance,
the comparison of means is carried out with a two-sided Welch's t-test.
This test will determine if the means of the two distributions are equal.
A large p-value implies different means, and thereby that one algorithm
yields smaller \(d_J\) than the other.
In this case, the algorithm with the smallest sample mean can be expected
to yield smaller \(d_J\) than the other, and thereby produce the shortest
path for the serial robot to travel.